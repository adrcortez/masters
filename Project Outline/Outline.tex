\documentclass[11pt, oneside]{article}   	% use "amsart" instead of "article" for AMSLaTeX format
\usepackage{geometry}                		% See geometry.pdf to learn the layout options. There are lots.
\geometry{letterpaper}                   		% ... or a4paper or a5paper or ... 
%\geometry{landscape}                		% Activate for for rotated page geometry
%\usepackage[parfill]{parskip}    		% Activate to begin paragraphs with an empty line rather than an indent
\usepackage{graphicx}				% Use pdf, png, jpg, or eps§ with pdflatex; use eps in DVI mode
								% TeX will automatically convert eps --> pdf in pdflatex		
\usepackage{amssymb}

\begin{document}

\begin{center}
    \huge
    \textbf{Spiral Packing}

    \vspace{0.1cm}
    \normalsize
    \textbf{Final paper outline}
        
    \vspace{0.3cm}
    \large
    \textbf{Adrian Cortez}
    
    \vspace{0.1cm}
    \small
    \textbf{Rochester Institute of Technology}
    \vspace{0.3cm}
\end{center}

\subsection*{Abstract}
\begin{enumerate}
	\item Brief explanation of the problem.
	\item Statement of the purpose of the project.
	\item High-level overview of the results.
\end{enumerate}

\subsection*{Background}
\begin{enumerate}
	\item Brief introduction
	\begin{itemize}
		\item What is packing?
		\item What is spiral packing?
	\end{itemize}
	\item Problem description.
	\item Description of an Archimedean spiral.
	\begin{itemize}
		\item Mathematical definition of a spiral.
		\item Explanation of parameters.
		\item Defining the area of the spiral.
	\end{itemize}

	\item Spiral packing.
	\begin{itemize}
		\item Definition of terms - e.g. branching, spiral sets, parent versus child, etc..
		\item Listing the constraints to be followed when generating spiral sets. 
		\item Detailed explanation of the three geometric cases when branching.
	\end{itemize}
	
	\item Apollonian packing
	\begin{itemize}
		\item Description of circle packing
		\item Explanation of how spirals are approximated using circles.
		\item Explanation of how the  spiral packing problem is reduced to the circle packing problem using the approximation.
	\end{itemize}
	\item Discussion of related works.
\end{enumerate}

\subsection*{Implementation}
\begin{enumerate}
	\item Description of the general algorithm.
	\item Explanation of how circle approximations are used during each branching case.
	\item Pseudocode of the algorithm.
	\item Description of the final program.
	\begin{itemize}
		\item What parts are automated?
		\item What parts require user input?
		\item Steps the user has to take to generate an image.
	\end{itemize}
\end{enumerate}

\subsection*{Results}
\begin{enumerate}
	\item Generated images labeled with porosity.
\end{enumerate}

\subsection*{Evaluation}
\begin{enumerate}
	\item Discussion of how the algorithm is evaluated.
	\item Discussion of the visual aspects of the images.
	\begin{itemize}
		\item Do the images look good?
		\item Do the images look packed?
		\item Are there any anomalies or errors?
		\item Are there any packing configurations or shape boundaries that don't work well?
	\end{itemize}
	\item Calculating the porosity and discussion around the packing efficiency.
\end{enumerate}

\subsection*{Discussion}
\begin{enumerate}
	\item Overall discussion of the spiral packing algorithm.
	\item Discussion of how well circle approximations worked.
	\item Discussion of results
	\begin{itemize}
		\item Explanation of visual anomalies and illusionary effects.
		\item Comments on the packing efficiency and porosity.
	\end{itemize}
\end{enumerate}
\end{document}  